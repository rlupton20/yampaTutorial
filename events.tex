One of the nice properties of Yampa is that it can handle continuous and discrete signals (which are realised as a type of continuous signal) in one framework. A discrete signal is thought of as a signal whose value is either ``Nothing happened'', or ``Something happened, and here is some information''. More prescisely, a discrete signal is a signal with values of type \hask{Event a} where

\begin{lstlisting}
data Event a = NoEvent | Event a
\end{lstlisting}

Examples of discrete events might be mouseclicks or key presses, or when the ambient temperature exceeds a certain level.

We refer to signals of \hask{Event} type as \emph{event streams}. A signal function of type \hask{SF a (Event b)} is called an \emph{event source}.

A sample of an event stream will take two (types of) value: either \hask{NoEvent} or \hask{Event a} for some value of type \hask{a}. However, events should not occur with infinite density on a signal, and indeed, the sampling rate provides an upper bound for how often events can occur on a signal. If this is a problem, one should implement some buffering for events.

\section{Switching and events}

Events are used to initiate changes in signal functions.
