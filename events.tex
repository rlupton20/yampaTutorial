One of the nice properties of Yampa is that it can handle continuous and discrete signals (which are realised as a type of continuous signal) in one framework. A discrete signal is thought of as a signal whose value is either ``Nothing happened'', or ``Something happened, and here is some information''. More prescisely, a discrete signal is a signal with values of type \hask{Event a} where

\begin{lstlisting}
data Event a = NoEvent | Event a
\end{lstlisting}

Examples of discrete events might be mouseclicks or key presses, or when the ambient temperature exceeds a certain level.

We refer to signals of \hask{Event} type as \emph{event streams}. A signal function of type \hask{SF a (Event b)} is called an \emph{event source}.

A sample of an event stream will take two (types of) value: either \hask{NoEvent} or \hask{Event a} for some value of type \hask{a}. However, events should not occur with infinite density on a signal, and indeed, the sampling rate provides an upper bound for how often events can occur on a signal. If this is a problem, one should implement some buffering for events.

\section{Switching and events}

Events are used to initiate changes in signal functions. Yampa provides combinators which allow us to describe switching in one signal function for another when an event occurs. The most basic of these is \hask{switch}:

\begin{lstlisting}
switch :: SF a (b, Event c) -> (c -> SF a b) -> SF a b
\end{lstlisting}

\noindent The first argument of switch is best thought of as taking the form \hask{(sigFun \&\&\& eventSource)}, where \hask{sigFun :: SF a b} and \hask{eventSource :: SF a (Event c)}. The idea is that \hask{switch} says to use the signal function \hask{sigFun}, until \hask{eventSource} supplies an event. The event will be a sample of type \hask{Event c}, and this wrapped value of type \hask{c} is used to determine what the replacement signal function should be --- this is what the second input of type \hask{c -> SF a b} does. The result is a signal function of type \hask{SF a b}.

In a more intuitive language, \hask{sigFun} is switched for a different signal function when an event occurs.
