One of the nice properties of Yampa is that it can handle continuous and discrete signals (which are realised as a type of continuous signal) in one framework. A discrete signal is thought of as a signal whose value is either ``Nothing happened'', or ``Something happened, and here is some information''. More prescisely, a discrete signal is a signal with values of type \hask{Event a} where

\begin{lstlisting}
data Event a = NoEvent | Event a
\end{lstlisting}

Examples of discrete events might be mouseclicks or key presses, or when the ambient temperature exceeds a certain level.

We refer to signals of \hask{Event} type as \emph{event streams}. A signal function of type \hask{SF a (Event b)} is called an \emph{event source}.

A sample of an event stream will take two (types of) value: either \hask{NoEvent} or \hask{Event a} for some value of type \hask{a}. However, events should not occur with infinite density on a signal, and indeed, the sampling rate provides an upper bound for how often events can occur on a signal. If this is a problem, one should implement some buffering for events.

\section{Switching and events}

Events are used to initiate changes in signal functions. Yampa provides combinators which allow us to describe switching in one signal function for another when an event occurs. The most basic of these is \hask{switch}:

\begin{lstlisting}
switch :: SF a (b, Event c) -> (c -> SF a b) -> SF a b
\end{lstlisting}

\noindent The first argument of switch is best thought of as taking the form \hask{(sigFun \&\&\& eventSource)}, where \hask{sigFun :: SF a b} and \hask{eventSource :: SF a (Event c)}. The idea is that \hask{switch} says to use the signal function \hask{sigFun}, until \hask{eventSource} supplies an event. The event will be a sample of type \hask{Event c}, and this wrapped value of type \hask{c} is used to determine what the replacement signal function should be --- this is what the second input of type \hask{c -> SF a b} does. The result is a signal function of type \hask{SF a b}.

In a more intuitive language, \hask{sigFun} is switched for a different signal function when an event occurs. Let's provide an example. We will rewrite our first hello program to use a switch to output a \hask{True} signal when a time of two seconds elapses.

\lstinputlisting[caption={HelloEvent.hs}, label=lst:template]{./src/HelloEvent.hs}

Our new version is certainly no simpler than the original, but it demonstrates basic use of events. \hask{waitTwo} is still our signal function, but now it makes use of a \hask{switch} to change its output. First \hask{waitTwo} outputs a constant signal false. When an event occurs from \hask{twoElapsed} however, it takes the value from this event (in this case \hask{()}) and uses the passed function to decide what signal function to switch to. In our case, the supplied function has a constant value \hask{constant True}, so upon an event from \hask{twoElapsed}, our signal function changes to take the value \hask{True} constantly.

So now the timing work must be done in \hask{twoElapsed}. This looks much the same as before, except now \hask{time >>> arr (>=2)} is fed into a new signal function \hask{edge}. \hask{edge} has type \hask{SF Bool (Event ())}. \hask{edge} takes an input signal of value type \hask{Bool}, and when the signal value changes (from \hask{True} to \hask{False}, or \hask{False} to \hask{True}), it ouputs an \hask{Event ()} on its output; it otherwise outputs \hask{NoEvent}. In sum then, this means an event is generated when (at least) two seconds have elapsed.

The sum effect is that our program prints its conclusion after waiting for two seconds. Notice that \hask{switch} is written in infix notation. One of the benefits of this is that it the initial signal function occurs on the left of \hask{switch} and the replacement upon an event on the right.

\begin{exercise}
When a signal function \hask{sf} is switched for a signal function \hask{sf'}, the time for \hask{sf'} starts again from zero. In other words signal functions have their own local time. Write a small program to demonstrate this.
\end{exercise}
